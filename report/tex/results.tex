% ______________________________________________________________________________
%
% DVG001 -- Introduktion till Linux och små nätverk
%                              Inlämningsuppgift #4
% ~~~~~~~~~~~~~~~~~~~~~~~~~~~~~~~~~~~~~~~~~~~~~~~~~
% Author:   Jonas Sjöberg
%           tel12jsg@student.hig.se
%
% Date:     2016-04-06 -- 2016-04-11
%
% License:  Creative Commons Attribution 4.0 International (CC BY 4.0)
%           <http://creativecommons.org/licenses/by/4.0/legalcode>
%           See LICENSE.md for additional licensing information.
% ______________________________________________________________________________


\section{Resultat}
Laborationen har demonstrerat flera viktiga grundläggande koncept i
\texttt{IP}-protokollet och nätverksadministration i Debian.

% ~~~~~~~~~~~~~~~~~~~~~~~~~~~~~~~~~~~~~~~~~~~~~~~~~~~~~~~~~~~~~~~~~~~~~~~~~~~~~~
\section{Diskussion}
Det här är den första laborationen som jag verkligen behövt anstränga mig med
rent innehållsmässigt. Anledningen till detta är att jag aldrig har ägnat mycket
tid åt att administrera nätverk och de gånger de väl hänt så har enklare grafiska
gränssnitt ofta varit tillgängliga, varpå jag använt dem för att lösa uppgiften
med minst möjliga ansträngning. Detta skiljer sig markant från t.ex. shell-skript, 
som jag ägnat mycket tid åt och känner mig naturligt attraherad till.

Det är såklart mycket viktigt att ha bra koll på ''nätverk'' generellt och
nätverksadministration, särskilt i \texttt{UNIX}-liknande system, då det idag
är mycket vanligt förekommande i de allra flesta IT-sammanhang. Dessutom är en
stor del av internets infrastruktur baserat på \texttt{UNIX}-liknande system
och därför är kunskap och färdigheter som laborationen tagit upp väldigt
viktiga för allmänbildning i IT.


% ~~~~~~~~~~~~~~~~~~~~~~~~~~~~~~~~~~~~~~~~~~~~~~~~~~~~~~~~~~~~~~~~~~~~~~~~~~~~~~
\section{Slutsatser}
Jag känner att jag förstår laborationens innehåll rent konceptuellt, men än för
verkligt intuitiv förståelse och djupare kunskap behövs fortsatt praktisering
av nätverksadministration.
