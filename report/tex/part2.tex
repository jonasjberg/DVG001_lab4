% ______________________________________________________________________________
%
% DVG001 -- Introduktion till Linux och små nätverk
%                              Inlämningsuppgift #4
% ~~~~~~~~~~~~~~~~~~~~~~~~~~~~~~~~~~~~~~~~~~~~~~~~~
% Author:   Jonas Sjöberg
%           tel12jsg@student.hig.se
%
% Date:     2016-04-06 -- 2016-04-11
%
% License:  Creative Commons Attribution 4.0 International (CC BY 4.0)
%           <http://creativecommons.org/licenses/by/4.0/legalcode>
%           See LICENSE.md for additional licensing information.
% ______________________________________________________________________________


\section{Del två}
Den här delen behandlar konfiguration nätverksinställningar med filen
''\texttt{/etc/network/interface}''.


% ______________________________________________________________________________
\subsection{Konfigurationsfilen ''\texttt{/etc/network/interface}''}
\subsubsection{Uppgiftsbeskrivning}
Här antar vi att ni har två maskiner, \textbf{m1} och \textbf{m2}, som är
anslutna till samma lokala nätverk.  Nätverket har en router, med namnet
\texttt{router}, med adressen \texttt{192.168.133.193/25}. Antag att maskinen
\textbf{m1} har nodadressen \texttt{10} och maskin \textbf{m2} har nodadressen
\texttt{20} i samma nät som routern ovan. Maskin \textbf{m1} skall ha statisk
inställning av nätverket \texttt{eth0} och maskinen \textbf{m2} skall ha
dynamisk (\texttt{dhcp}) inställning av nätverket på \texttt{eth0}.

Hur skulle ni skriva \texttt{/etc/network/interface} för respektive maskin
\textbf{m1} och \textbf{m2}?  Ange hur ni har kommit fram till innehållet i
filen.


\subsubsection{Lösning}
För att räkna ut nätadresserna för \textbf{m1} och \textbf{m2} används programmet
\texttt{ipcalc} enligt Programlistning~\ref{listing:sh_2-ipcalc}.
De första och sista möjliga adressen visas på rad 15 och 16 i Programlistning~\ref{listing:sh_2-ipcalc}:
\begin{verbatim}
HostMin: 192.168.133.129
HostMax: 192.168.133.254
\end{verbatim}
Det ger alltså $154 - 129 = 125$ möjliga adresser på nätet.

Nyckeln i lösningen av uppgiften är det faktum att maskin \texttt{m1} ska ha
adress nummer 10 bland alla sekventiellt ordnade möjliga adresser. Likaså ska
maskin \texttt{m2} ha den tjugonde adressen bland nätets möjliga adresser.

\begin{listing}[H]
  \shellcode{include/sh_2-ipcalc}
  \caption{Körning av programmet \texttt{ipcalc} för att räkna ut adresser.}
  \label{listing:sh_2-ipcalc}
\end{listing}


% Resultatet presenteras mer överskådligt i Tabell~\ref{table:interface-config}.
% \begin{table}[]
%   \centering
%   \caption{Lista över \texttt{MAC}- och \texttt{IP}-adresser för maskiner på
%            nätverket.}
%   \label{table:interface-config}
%   \begin{tabular}{@{}llll@{}}
%     \toprule
%     Enhet  & Nodadress                & Nätmask                  & \texttt{IP}-adress       \\ \midrule
%     router & \texttt{192.168.133.193} & \texttt{255.255.255.128} & \texttt{192.168.133.193} \\
%     m1     & \texttt{192.168.133.10}  & \texttt{255.255.255.128} & \texttt{192.168.133.193} \\
%     m2     & \texttt{192.168.133.20}  & \texttt{255.255.255.128} & \texttt{192.168.133.193} \\ \bottomrule
%   \end{tabular}
% \end{table}


\subsubsection{Konfigurationsfilerna \texttt{/etc/network/interface}}
Konfigurationsfilerna skrevs enligt \cite{debian:network} för att få formen
nedan.

För \textbf{m1} får konfigurationsfilen \texttt{/etc/network/interface}
utseendet i Programlistning~\ref{listing:sh_2-interface-m1}.  Och för
\textbf{m2} skulle motsvarande konfigurationsfil se ut som i
Programlistning~\ref{listing:sh_2-interface-m2}.


\begin{listing}[H]
  \shellcode{include/sh_2-interface-m1}
  \caption{Konfigurationsfil för \textbf{m1}.}
  \label{listing:sh_2-interface-m1}
\end{listing}

\begin{listing}[H]
  \shellcode{include/sh_2-interface-m2}
  \caption{Konfigurationsfil för \textbf{m2}.}
  \label{listing:sh_2-interface-m2}
\end{listing}
