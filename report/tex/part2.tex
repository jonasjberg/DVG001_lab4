% ______________________________________________________________________________
%
% DVG001 -- Introduktion till Linux och små nätverk
%                              Inlämningsuppgift #4
% ~~~~~~~~~~~~~~~~~~~~~~~~~~~~~~~~~~~~~~~~~~~~~~~~~
% Author:   Jonas Sjöberg
%           tel12jsg@student.hig.se
%
% Date:     2016-04-06 -- 2016-04-11
%
% License:  Creative Commons Attribution 4.0 International (CC BY 4.0)
%           <http://creativecommons.org/licenses/by/4.0/legalcode>
%           See LICENSE.md for additional licensing information.
% ______________________________________________________________________________


\section{Del två}
% TODO: introduktion till del 2


% ______________________________________________________________________________
\subsection{Konfigurationsfilen ''\texttt{/etc/network/interface}''}
\subsubsection{Uppgiftsbeskrivning}
Här antar vi att ni har två maskiner, \textbf{m1} och \textbf{m2}, som är
anslutna till samma lokala nätverk.  Nätverket har en router, med namnet
\texttt{router}, med adressen \texttt{192.168.133.193/25}. Antag att maskinen
\textbf{m1} har nodadressen \texttt{10} och maskin \textbf{m2} har nodadressen
\texttt{20} i samma nät som routern ovan. Maskin \textbf{m1} skall ha statisk
inställning av nätverket \texttt{eth0} och maskinen \textbf{m2} skall ha
dynamisk (\texttt{dhcp}) inställning av nätverket på \texttt{eth0}.

Hur skulle ni skriva \texttt{/etc/network/interface} för respektive maskin
\textbf{m1} och \textbf{m2}?  Ange hur ni har kommit fram till innehållet i
filen.

router   192.168.133.193/25

Problembeskrivningen återges i Tabell~\ref{table:network}.

\begin{table}[]
  \centering
  \caption{Lista över \texttt{MAC}- och \texttt{IP}-adresser för maskiner på
           nätverket.}
  \label{table:network}
  \begin{tabular}{@{}llll@{}}
    \toprule
    Enhet   & \texttt{IP}-adress     & \texttt{MAC}-address       & Enhetens tillverkare          \\ \midrule
    router  & \texttt{192.168.133.193/25} & 
    m1      & \texttt{192.168.1.1}   & \texttt{04:EC:08:A7:8B:A0} & Tp-link Technologies CO.      \\
    m2      & \texttt{192.168.1.101} & \texttt{08:9E:0C:A4:50:30} & Apple                         \\
    \texttt{192.168.1.110} & \texttt{04:11:0B:37:FC:90} & Hewlett Packard               \\ \bottomrule
  \end{tabular}
\end{table}

\subsubsection{Lösning}
% TODO: Beskriv lösningen av uppgiften ..
