% ______________________________________________________________________________
%
% DVG001 -- Introduktion till Linux och små nätverk
%                              Inlämningsuppgift #3
% ~~~~~~~~~~~~~~~~~~~~~~~~~~~~~~~~~~~~~~~~~~~~~~~~~
% Author:   Jonas Sjöberg
%           tel12jsg@student.hig.se
%
% Date:     2016-03-15 -- 2016-03-20
%
% License:  Creative Commons Attribution 4.0 International (CC BY 4.0)
%           <http://creativecommons.org/licenses/by/4.0/legalcode>
%           See LICENSE.md for additional licensing information.
% ______________________________________________________________________________


\section{Inledning}\label{inledning}
% Skriv en kort inledning här som beskriver kortfattat vad rapporten handlar
% om. Den skall vara orienterande om Bakgrund och Syfte.
Laborationen behandlar hantering av användare och grupper, rättigheter och
filåtkomst, installation och konfiguration av program, behandling av kataloger
och filer, samt visualisering av filsystemet med skapandet av en trädstruktur.
En del pakethantering med installation och konfiguration av program kommer
också att behandlas, särskilt tidssynkronisering med \texttt{ntd} och körning
av program och kommandon med administratörsrättigheter med \texttt{sudo}.


% ______________________________________________________________________________
\subsection{Bakgrund}
%    Beskriv lite mer ingående om bakgrunden till uppgiften, vad den handlar om.
Laborationen bygger vidare på de föregående laborationerna och behandlar vidare
grundläggande koncept i administation av \texttt{UNIX}-liknande system.

Den virtuella maskin som skapades tidigare under kursens gång används under
laborationen.

% ______________________________________________________________________________
\subsection{Syfte}
% Skriv lite mer ingående om syftet med uppgiften.
Syftet med laborationen är att demonstrera och öva på systemadministration,
däribland hantering av rättigheter för användare, grupper, filer och kataloger.


% ______________________________________________________________________________
\subsection{Arbetsmetod}
% Hur kommer ni att arbeta?  Detta är en lite längre text än den rent
% orienterande texten i Planering och genomförande ovan.

Nedan följer en preliminär redogörelse för den experimentuppställning som används
under laborationen:

\begin{itemize}
  \item Laborationen utförs på en \texttt{ProBook-6545b} laptop som kör
        \texttt{Xubuntu 15.10} på kerneln \texttt{Linux 3.19.0-28}.

  \item Rapporten skrivs i \LaTeX\  som kompileras till pdf med \texttt{latexmk}.
        Detta sker på värdsystemet.

  \item Virtualisering sker med \texttt{Oracle VirtualBox} version
        \texttt{5.0.10\_Ubuntu r104061}.

  \item Utveckling av programkod och testkörning sker i gästsystemet som kör
        \texttt{Debian 7.3 (jessie)} på kerneln \texttt{Linux 3.16.0-4}.

  \item Både rapporten och koden skrivs med texteditorn \texttt{Vim}.

  \item För versionshantering av både rapporten och programkod används \texttt{Git}.
    \begin{itemize}
      \item Källkod till programmet och rapporten finns att hämta på:

            \url{https://github.com/jonasjberg/DVG001\_lab3}

      \item Hämta hem repon genom att exekvera följande från kommandoraden:
            
            \texttt{git clone git@github.com:jonasjberg/DVG001\_lab3.git}

    \end{itemize}
\end{itemize}


