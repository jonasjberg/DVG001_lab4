% ______________________________________________________________________________
%
% DVG001 -- Introduktion till Linux och små nätverk
%                              Inlämningsuppgift #4
% ~~~~~~~~~~~~~~~~~~~~~~~~~~~~~~~~~~~~~~~~~~~~~~~~~
% Author:   Jonas Sjöberg
%           tel12jsg@student.hig.se
%
% Date:     2016-04-06 -- 2016-04-11
%
% License:  Creative Commons Attribution 4.0 International (CC BY 4.0)
%           <http://creativecommons.org/licenses/by/4.0/legalcode>
%           See LICENSE.md for additional licensing information.
% ______________________________________________________________________________


\section{Del ett}
% TODO: introduktion till del 1


% ______________________________________________________________________________
\subsection{\texttt{IP}-nummer}
\subsubsection{Uppgiftsbeskrivning}
Här är uppgiften att ta reda på vilken \texttt{IP}-adress som maskinen har samt
hur den fått sin adress genom att använda kommandot \texttt{ip} och titta i
loggar, exempelvis \texttt{/var/log/messages}.


\subsubsection{Lösning}
För att ta reda på datorns \texttt{IP}-adress körs kommandot i
Programlistning~\ref{listing:sh_1-ip}.

\begin{listing}[H]
  \shellcode{include/sh_1-ip}
  \caption{Kommando för att ta reda på datorns \texttt{IP}-adress.}
  \label{listing:sh_1-ip}
\end{listing}

Resultatet visar att datorns \texttt{IP}-adress är \texttt{192.168.1.112}
med nätmasken \texttt{24}.

Ett shell-skript används för att söka igenom loggar efter den aktuella
\texttt{IP}-adressen. Programmet letar efter filer i sökvägen \texttt{/var/log}
och undersöker filernas innehåll genom att läsa ''magic header bytes'' som
avgör vilken typ av fil det är. Om en hittad fil är en vanlig textfil söks dess
innehåll efter datorns aktuella \texttt{IP}-adress som extraherats tidigare.

Programmet visas i Programlistning~\ref{listing:sh_1-grep-logs.sh} och
resultatet av en körning visas i
Figur~\ref{fig:sh_1-grep-logs.sh_output}

\begin{listing}[H]
  \shellcode{include/sh_1-grep-logs.sh}
  \caption{Kommando för att söka igenom loggar i sökvägen \texttt{/var/log}
           efter omnämnanden av datorns aktuella \texttt{IP}-adress.}
  \label{listing:sh_1-grep-logs.sh}
\end{listing}

\begin{figure}[htp]
  \centering
  \includegraphics[scale=0.85]{include/sh_1-grep-logs-sh_output.pdf}
  \caption{Körning av programmet i Programlistning~\ref{listing:sh_1-grep-logs.sh}.}
  \label{fig:sh_1-grep-logs.sh_output}
\end{figure}

Bland resultaten finns omnämnanden av \texttt{mDNS} som rör \texttt{DHCP}.
Instruktionerna nämner också att den grafiska miljön har verktyg för att
konfigurera nätverksinställningar, en skärmdump på detta visas i
Figur~\ref{fig:scr_1-network_A} och Figur~\ref{fig:scr_1-network_B} .

\begin{figure}[htp]
  \centering
  \includegraphics[scale=0.35]{include/scr_1-network_A.png}
  \caption{Nätverksinställningar i det grafiska gränssnittet.}
  \label{fig:scr_1-network_A}
\end{figure}

\begin{figure}[htp]
  \centering
  \includegraphics[scale=0.35]{include/scr_1-network_B.png}
  \caption{Nätverksinställningar i det grafiska gränssnittet.}
  \label{fig:scr_1-network_B}
\end{figure}


% ______________________________________________________________________________
\subsection{Nät- och Nodnummer}
\subsubsection{Uppgiftsbeskrivning}
Uppgiften är här att ange vilken nätadress man har genom att använda
\texttt{CIDR} och dessutom ange nätmasken.


\subsubsection{Lösning}
För detaljerad information om datorns \texttt{IP}-adress körs kommandot i
Programlistning~\ref{listing:sh_1-ipcalc}. Resultatet av körningen visas i
Figur~\ref{fig:sh_1-ipcalc_output}

\begin{listing}[H]
  \shellcode{include/sh_1-ipcalc}
  \caption{Kommando för att visa detaljerad information om en
           \texttt{IP}-adress.}
  \label{listing:sh_1-ipcalc}
\end{listing}

\begin{figure}[htp]
  \centering
  \includegraphics[scale=0.85]{include/sh_1-ipcalc_output.pdf}
  \caption{Körning av programmet i Programlistning~\ref{listing:sh_1-ipcalc}.}
  \label{fig:sh_1-ipcalc_output}
\end{figure}



% ______________________________________________________________________________
\subsection{Routeradresser}
\subsubsection{Uppgiftsbeskrivning}
Uppgiften är att lista de routrar som maskinen känner till och dessutom ange
vilken som är standardroutern.

\subsubsection{Lösning}
För att få en lista på routrar som datorn känner till används kommandot \texttt{ip route list}.
Körning med resultat visas i Programlistning~\ref{listing:sh_1-ip-route-list}.

\begin{listing}[H]
  \shellcode{include/sh_1-ip-route-list}
  \caption{Körning av kommando för att lista information om routers på nätverk.}
  \label{listing:sh_1-ip-route-list}
\end{listing}

Resultatet visar att standard-routern har \texttt{IP}-adress \texttt{192.168.1.1}.

Lite mer detaljerad men i stort sett samma information kan fås med \texttt{ip
neighbor} som i Programlistning~\ref{listing:sh_1-ip-neighbor}. 

\begin{listing}[H]
  \shellcode{include/sh_1-ip-neighbor}
  \caption{Körning av kommando för att lista information maskiner i samma nät.}
  \label{listing:sh_1-ip-neighbor}
\end{listing}

Anmärkningsvärt är att datorn rapporterar att den enhet som används är av typen
\texttt{eth0}.  Men eftersom att datorn körs som en virtuell maskin på en
laptop som ansluter till nätverket genom en trådlös WIFI-anslutning så är det
uppenbarligen inte sant.  Den trådbundna anslutningen är virtuell och skapas av
\texttt{VirtualBox}.

% ______________________________________________________________________________
\subsectionM{\texttt{MAC}-adresser}
\subsubsection{Uppgiftsbeskrivning}
Här är uppgiften att lista \texttt{MAC}- och \texttt{IP}-adresser för alla
maskiner i nätverket.

\subsubsection{Lösning}
% TODO: Beskriv lösningen av uppgiften ..


