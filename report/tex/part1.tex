% ______________________________________________________________________________
%
% DVG001 -- Introduktion till Linux och små nätverk
%                              Inlämningsuppgift #4
% ~~~~~~~~~~~~~~~~~~~~~~~~~~~~~~~~~~~~~~~~~~~~~~~~~
% Author:   Jonas Sjöberg
%           tel12jsg@student.hig.se
%
% Date:     2016-04-06 -- 2016-04-11
%
% License:  Creative Commons Attribution 4.0 International (CC BY 4.0)
%           <http://creativecommons.org/licenses/by/4.0/legalcode>
%           See LICENSE.md for additional licensing information.
% ______________________________________________________________________________


\section{Del ett}
% TODO: introduktion till del 1


% ______________________________________________________________________________
\subsection{\texttt{IP}-nummer}
\subsubsection{Uppgiftsbeskrivning}
Här är uppgiften att ta reda på vilken \texttt{IP}-adress som maskinen har samt
hur den fått sin adress genom att använda kommandot \texttt{ip} och titta i
loggar, exempelvis \texttt{/var/log/messages}.


\subsubsection{Lösning}
% TODO: Beskriv lösningen av uppgiften ..


% ______________________________________________________________________________
\subsection{Nät- och Nodnummer}
\subsubsection{Uppgiftsbeskrivning}
Uppgiften är här att ange vilken nätadress man har genom att använda
\texttt{CIDR} och dessutom ange nätmasken.


\subsubsection{Lösning}
% TODO: Beskriv lösningen av uppgiften ..


% ______________________________________________________________________________
\subsection{Routeradresser}
\subsubsection{Uppgiftsbeskrivning}
Uppgiften är att lista de routrar som maskinen känner till och dessutom ange
vilken som är standardroutern.

\subsubsection{Lösning}
% TODO: Beskriv lösningen av uppgiften ..


% ______________________________________________________________________________
\subsectionM{\texttt{MAC}-adresser}
\subsubsection{Uppgiftsbeskrivning}
Här är uppgiften att lista \texttt{MAC}- och \texttt{IP}-adresser för alla 
maskiner i nätverket.

\subsubsection{Lösning}
% TODO: Beskriv lösningen av uppgiften ..


