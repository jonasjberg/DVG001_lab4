% ______________________________________________________________________________
%
% DVG001 -- Introduktion till Linux och små nätverk
%                              Inlämningsuppgift #4
% ~~~~~~~~~~~~~~~~~~~~~~~~~~~~~~~~~~~~~~~~~~~~~~~~~
% Author:   Jonas Sjöberg
%           tel12jsg@student.hig.se
%
% Date:     2016-04-06 -- 2016-04-11
%
% License:  Creative Commons Attribution 4.0 International (CC BY 4.0)
%           <http://creativecommons.org/licenses/by/4.0/legalcode>
%           See LICENSE.md for additional licensing information.
% ______________________________________________________________________________

\documentclass[11pt,a4paper]{article}

\usepackage[utf8]{inputenc}
\inputencoding{utf8}
\usepackage[swedish]{babel}
\usepackage[swedish]{isodate}
\usepackage[T1]{fontenc}

\usepackage{lmodern}
\usepackage{fullpage}

\usepackage{csquotes}               % Behövs av biblatex

\usepackage[natbib=true,
            style=ieee,
            backend=biber]{biblatex}
\addbibresource{tex/refs.bib}

\usepackage[binary-units=true]{siunitx}
\usepackage{float}
\usepackage{textcomp}
\usepackage{url}
\usepackage{graphicx}
%\usepackage{amssymb}
%\usepackage{amsmath}
\usepackage{amsfonts}
\usepackage{graphicx}
%\usepackage{microtype}

\usepackage[pdfusetitle,
            bookmarks=true,
            bookmarksnumbered=true,
            bookmarksopen=false,
            breaklinks=false,
            pdfborder={0 0 0},
            backref=false,
            colorlinks=false,
            hidelinks]{hyperref}

\newcommand{\screenshot}[4]{
\begin{figure}[H]
\centering
%\includegraphics[width=\linewidth]{#1}
\includegraphics[height=8.0cm]{#1}
\caption[#2]{#3}
\label{#4}
\end{figure}
}

\usepackage{minted}
\usemintedstyle{bw}

\usepackage{verbatim}
\usepackage{fancyvrb}
\usepackage{listings}

\newmintedfile[shellcode]{bash}{
%bgcolor=mintedbackground,
%fontfamily=tt,
fontsize=\footnotesize,
linenos=true,
numberblanklines=true,
numbersep=12pt,
numbersep=5pt,
%gobble= 0,
frame= lines,
%framerule= 0.4pt,
framesep=2mm,
funcnamehighlighting=true,
tabsize=4,
obeytabs=false,
mathescape=false
samepage=false,
showspaces=false,
showtabs=false,
texcl=false,
}

\newmintedfile[configfile]{linux-config}{
%bgcolor=mintedbackground,
%fontfamily=tt,
fontsize=\footnotesize,
linenos=true,
numberblanklines=true,
numbersep=12pt,
numbersep=5pt,
%gobble=0,
frame=lines,
%framerule=0.4pt,
framesep=2mm,
funcnamehighlighting=true,
tabsize=4,
obeytabs=false,
mathescape=false
samepage=false,
showspaces=false,
showtabs=false,
texcl=false,
}


\expandafter\def\csname PY@tok@err\endcsname{}
\expandafter\def\csname PYGdefault@tok@err\endcsname{\def\PYGdefault@bc##1{{\strut ##1}}}

\renewcommand\listingscaption{Programlistning}
\renewcommand\listoflistingscaption{Programlistningar}

\usepackage{booktabs}
\usepackage{longtable}

\usepackage{pdfpages}


\title{\textsc{DVG001}                         \\
       Introduktion till Linux och små nätverk \\
       Laboration 4}

\author{                                 \\
  Jonas Sjöberg                          \\
  860224-xxxx                            \\
  Högskolan i Gävle                      \\
  \texttt{tel12jsg@student.hig.se}       \\
  \texttt{https://github.com/jonasjberg} \\
}

\date{}

\begin{document}
  \maketitle

  \begin{center}
    \begin{tabular}{l r}
      Utförd: & \isodate \printdate{2016-04-06} -- \printdate{2016-04-11} \\
      Korrigerad: & \isodate \printdate{2016-04-29}                       \\
      Kursansvarig lärare: & Anders Jackson                               \\
                           & Anders Hermansson
    \end{tabular}
  \end{center}

  \begin{abstract}
    Laboration i kursen \emph{DVG001 -- Introduktion till Linux och små
    nätverk} som läses på distans via Högskolan i Gävle under vårterminen 2016.
    Laborationen behandlar grundläggande koncept i nätverk,
    \texttt{IP}-protokollet och nätverksadministration.
  \end{abstract}

  \newpage
  %\hypersetup{linkcolor=black}
  \setcounter{tocdepth}{3}
  \tableofcontents

  \bigskip

  \listoffigures
  \listoftables
  \listoflistings

  %\expandafter\def\csname PY@tok@err\endcsname{}

  \newpage
  % ______________________________________________________________________________
%
% DVG001 -- Introduktion till Linux och små nätverk
%                              Inlämningsuppgift #4
% ~~~~~~~~~~~~~~~~~~~~~~~~~~~~~~~~~~~~~~~~~~~~~~~~~
% Author:   Jonas Sjöberg
%           tel12jsg@student.hig.se
%
% Date:     2016-04-06 -- 2016-04-11
%
% License:  Creative Commons Attribution 4.0 International (CC BY 4.0)
%           <http://creativecommons.org/licenses/by/4.0/legalcode>
%           See LICENSE.md for additional licensing information.
% ______________________________________________________________________________


\section{Inledning}
% Skriv en kort inledning här som beskriver kortfattat vad rapporten handlar
% om. Den skall vara orienterande om Bakgrund och Syfte.


% ______________________________________________________________________________
\subsection{Bakgrund}
%    Beskriv lite mer ingående om bakgrunden till uppgiften, vad den handlar om.
Laborationen bygger vidare på de föregående laborationerna och behandlar vidare
grundläggande aspekter av kommunikation mellan datorer i ett nätverk.

Den virtuella maskin som skapades tidigare under kursens gång används under
laborationen.

% ______________________________________________________________________________
\subsection{Syfte}
% Skriv lite mer ingående om syftet med uppgiften.
Syftet med laborationen är att demonstrera och ge tillfälle till övning på
systemadministration, särskilt relaterat till nätverk.

% ______________________________________________________________________________
\subsection{Arbetsmetod}
% Hur kommer ni att arbeta?  Detta är en lite längre text än den rent
% orienterande texten i Planering och genomförande ovan.

Nedan följer en preliminär redogörelse för den experimentuppställning som används
under laborationen:

\begin{itemize}
  \item Laborationen utförs på en \texttt{ProBook-6545b} laptop som kör
        \texttt{Xubuntu 15.10} på kerneln \texttt{Linux 3.19.0-28}.

  \item Rapporten skrivs i \LaTeX\  som kompileras till pdf med \texttt{latexmk}.
        Detta sker på värdsystemet.

  \item Virtualisering sker med \texttt{Oracle VirtualBox} version
        \texttt{5.0.10\_Ubuntu r104061}.

  \item Utveckling av programkod och testkörning sker i gästsystemet som kör
        \texttt{Debian 7.3 (jessie)} på kerneln \texttt{Linux 3.16.0-4}.

  \item Både rapporten och koden skrivs med texteditorn \texttt{Vim}.

  \item För versionshantering av både rapporten och programkod används \texttt{Git}.
    \begin{itemize}
      \item Källkod till programmet och rapporten finns att hämta på:

            \url{https://github.com/jonasjberg/DVG001\_lab4}

      \item Hämta hem repon genom att exekvera följande från kommandoraden:
            
            \texttt{git clone git@github.com:jonasjberg/DVG001\_lab4.git}

    \end{itemize}
\end{itemize}



  % ______________________________________________________________________________
%
% DVG001 -- Introduktion till Linux och små nätverk
%                              Inlämningsuppgift #4
% ~~~~~~~~~~~~~~~~~~~~~~~~~~~~~~~~~~~~~~~~~~~~~~~~~
% Author:   Jonas Sjöberg
%           tel12jsg@student.hig.se
%
% Date:     2016-04-06 -- 2016-04-11
%
% License:  Creative Commons Attribution 4.0 International (CC BY 4.0)
%           <http://creativecommons.org/licenses/by/4.0/legalcode>
%           See LICENSE.md for additional licensing information.
% ______________________________________________________________________________


\section{Genomförande}
% Här skriver ni vilka steg ni gjorde och resultatet av dem. Ni skall ha med
% information så att vi kan se hur ni har gjort, dvs. beskrivande text,
% skärmdumpar, bilder etc.  Längre skärmdumpar, innehåll i relevanta filer och
% större bilder lägger ni i bilagor, som bilaga I, så att de inte tar över en
% sida själva.
% Kommandon som ni skriver i ett skal skall skrivas i detta format, som är
% teckenformatmall ”Exempel” i OpenOffice/LibreOffice. Detta så att de
% skiljer sig från övriga brödtext i stycket.  Detta underlättar läsningen för
% andra, som oss lärare.







  % ______________________________________________________________________________
%
% DVG001 -- Introduktion till Linux och små nätverk
%                              Inlämningsuppgift #4
% ~~~~~~~~~~~~~~~~~~~~~~~~~~~~~~~~~~~~~~~~~~~~~~~~~
% Author:   Jonas Sjöberg
%           tel12jsg@student.hig.se
%
% Date:     2016-04-06 -- 2016-04-11
%
% License:  Creative Commons Attribution 4.0 International (CC BY 4.0)
%           <http://creativecommons.org/licenses/by/4.0/legalcode>
%           See LICENSE.md for additional licensing information.
% ______________________________________________________________________________


\section{Del ett}
% TODO: introduktion till del 1


% ______________________________________________________________________________
\subsection{\texttt{IP}-nummer}
\subsubsection{Uppgiftsbeskrivning}
Här är uppgiften att ta reda på vilken \texttt{IP}-adress som maskinen har samt
hur den fått sin adress genom att använda kommandot \texttt{ip} och titta i
loggar, exempelvis \texttt{/var/log/messages}.


\subsubsection{Lösning}
% TODO: Beskriv lösningen av uppgiften ..
För att ta reda på datorns \texttt{IP}-adress körs kommandot i
Programlistning~\ref{listing:sh_1-ip}.

\begin{listing}[H]
\shellcode{tex/sh_1-ip}
\caption{Kommando för att ta reda på datorns \texttt{IP}-adress.}
\label{listing:sh_1-ip}
\end{listing}

Resultatet visar att datorns \texttt{IP}-adress är \texttt{192.168.1.112} 
med nätmasken \texttt{24}.

% ______________________________________________________________________________
\subsection{Nät- och Nodnummer}
\subsubsection{Uppgiftsbeskrivning}
Uppgiften är här att ange vilken nätadress man har genom att använda
\texttt{CIDR} och dessutom ange nätmasken.


\subsubsection{Lösning}
% TODO: Beskriv lösningen av uppgiften ..


% ______________________________________________________________________________
\subsection{Routeradresser}
\subsubsection{Uppgiftsbeskrivning}
Uppgiften är att lista de routrar som maskinen känner till och dessutom ange
vilken som är standardroutern.

\subsubsection{Lösning}
% TODO: Beskriv lösningen av uppgiften ..


% ______________________________________________________________________________
\subsectionM{\texttt{MAC}-adresser}
\subsubsection{Uppgiftsbeskrivning}
Här är uppgiften att lista \texttt{MAC}- och \texttt{IP}-adresser för alla 
maskiner i nätverket.

\subsubsection{Lösning}
% TODO: Beskriv lösningen av uppgiften ..



  % ______________________________________________________________________________
%
% DVG001 -- Introduktion till Linux och små nätverk
%                              Inlämningsuppgift #4
% ~~~~~~~~~~~~~~~~~~~~~~~~~~~~~~~~~~~~~~~~~~~~~~~~~
% Author:   Jonas Sjöberg
%           tel12jsg@student.hig.se
%
% Date:     2016-04-06 -- 2016-04-11
%
% License:  Creative Commons Attribution 4.0 International (CC BY 4.0)
%           <http://creativecommons.org/licenses/by/4.0/legalcode>
%           See LICENSE.md for additional licensing information.
% ______________________________________________________________________________


\section{Del två}
Den här delen behandlar konfiguration nätverksinställningar med filen
''\texttt{/etc/network/interface}''.


% ______________________________________________________________________________
\subsection{Konfigurationsfilen ''\texttt{/etc/network/interface}''}
\subsubsection{Uppgiftsbeskrivning}
Här antar vi att ni har två maskiner, \textbf{m1} och \textbf{m2}, som är
anslutna till samma lokala nätverk.  Nätverket har en router, med namnet
\texttt{router}, med adressen \texttt{192.168.133.193/25}. Antag att maskinen
\textbf{m1} har nodadressen \texttt{10} och maskin \textbf{m2} har nodadressen
\texttt{20} i samma nät som routern ovan. Maskin \textbf{m1} skall ha statisk
inställning av nätverket \texttt{eth0} och maskinen \textbf{m2} skall ha
dynamisk (\texttt{dhcp}) inställning av nätverket på \texttt{eth0}.

Hur skulle ni skriva \texttt{/etc/network/interface} för respektive maskin
\textbf{m1} och \textbf{m2}?  Ange hur ni har kommit fram till innehållet i
filen.


\subsubsection{Lösning}
För att räkna ut nätadresserna för \textbf{m1} och \textbf{m2} används programmet
\texttt{ipcalc} enligt Programlistning~\ref{listing:sh_2-ipcalc}.
Resultatet presenteras mer överskådligt i Tabell~\ref{table:interface-config}.


\begin{listing}[H]
  \shellcode{include/sh_2-ipcalc}
  \caption{Körning av programmet \texttt{ipcalc} för att räkna ut adresser.}
  \label{listing:sh_2-ipcalc}
\end{listing}


\begin{table}[]
  \centering
  \caption{Lista över \texttt{MAC}- och \texttt{IP}-adresser för maskiner på
           nätverket.}
  \label{table:interface-config}
  \begin{tabular}{@{}llll@{}}
    \toprule
    Enhet  & Nodadress                & Nätmask                  & \texttt{IP}-adress          \\ \midrule
    router & \texttt{192.168.133.193} & \texttt{255.255.255.128} & \texttt{192.168.133.193} \\
    m1     & \texttt{192.168.133.10}  & \texttt{255.255.255.128} & \texttt{0}        \\
    m2     & \texttt{192.168.133.20}  & \texttt{255.255.255.128} & \texttt{0}      \\ \bottomrule
  \end{tabular}
\end{table}


\subsubsection{Konfigurationsfilerna \texttt{/etc/network/interface}}
Konfigurationsfilerna skrevs enligt \cite{debian:network} för att få formen
nedan.

För \textbf{m1} får konfigurationsfilen \texttt{/etc/network/interface}
utseendet i Programlistning~\ref{listing:sh_2-interface-m1}.  Och för
\textbf{m2} skulle motsvarande konfigurationsfil se ut som i
Programlistning~\ref{listing:sh_2-interface-m2}.


\begin{listing}[H]
  \shellcode{include/sh_2-interface-m1}
  \caption{Konfigurationsfil för \textbf{m1}.}
  \label{listing:sh_2-interface-m1}
\end{listing}

\begin{listing}[H]
  \shellcode{include/sh_2-interface-m2}
  \caption{Konfigurationsfil för \textbf{m2}.}
  \label{listing:sh_2-interface-m2}
\end{listing}


  \newpage
  % ______________________________________________________________________________
%
% DVG001 -- Introduktion till Linux och små nätverk
%                              Inlämningsuppgift #4
% ~~~~~~~~~~~~~~~~~~~~~~~~~~~~~~~~~~~~~~~~~~~~~~~~~
% Author:   Jonas Sjöberg
%           tel12jsg@student.hig.se
%
% Date:     2016-04-06 -- 2016-04-11
%
% License:  Creative Commons Attribution 4.0 International (CC BY 4.0)
%           <http://creativecommons.org/licenses/by/4.0/legalcode>
%           See LICENSE.md for additional licensing information.
% ______________________________________________________________________________


\section{Resultat}
Laborationen har demonstrerat flera viktiga grundläggande koncept i
\texttt{IP}-protokollet och nätverksadministration i Debian.

% ~~~~~~~~~~~~~~~~~~~~~~~~~~~~~~~~~~~~~~~~~~~~~~~~~~~~~~~~~~~~~~~~~~~~~~~~~~~~~~
\section{Diskussion}
Det här är den första laborationen som jag verkligen behövt anstränga mig med
rent innehållsmässigt. Anledningen till detta är att jag aldrig har ägnat mycket
tid åt att administrera nätverk och de gånger de väl hänt så har enklare grafiska
gränssnitt ofta varit tillgängliga, varpå jag använt dem för att lösa uppgiften
med minst möjliga ansträngning. Detta skiljer sig markant från t.ex. shell-skript, 
som jag ägnat mycket tid åt och känner mig naturligt attraherad till.

Det är såklart mycket viktigt att ha bra koll på ''nätverk'' generellt och
nätverksadministration, särskilt i \texttt{UNIX}-liknande system, då det idag
är mycket vanligt förekommande i de allra flesta IT-sammanhang. Dessutom är en
stor del av internets infrastruktur baserat på \texttt{UNIX}-liknande system
och därför är kunskap och färdigheter som laborationen tagit upp väldigt
viktiga för allmänbildning i IT.


% ~~~~~~~~~~~~~~~~~~~~~~~~~~~~~~~~~~~~~~~~~~~~~~~~~~~~~~~~~~~~~~~~~~~~~~~~~~~~~~
\section{Slutsatser}
Jag känner att jag förstår laborationens innehåll rent konceptuellt, men än för
verkligt intuitiv förståelse och djupare kunskap behövs fortsatt praktisering
av nätverksadministration.


  \addcontentsline{toc}{section}{Referenser}
  \printbibliography{}

\end{document}
